\section{Analysis}
    \paragraph{}
    The aim of this project is to create a scientific calculator with transcendental functions built from scratch. 7 potential users of this calculator were interviewed. And the interviews were based on 15 pre-determined questions.

    \paragraph{}
    It is clearly evident from the interviews that most of our interviewees prefer to have basic math functions like addition, subtraction, multiplication and division and, some transcendental complex functions like $arccos(x)$, $ab^x$, $log_b(x)$, $\gamma ()$, $MAD$ (Mean Absolute Deviation), $\sigma$, $sinh(x)$ and $x^y$ in the calculator. Hence, a scientific calculator with the above-mentioned functions would be the most appropriate calculator and would cater everyone’s needs.

    \paragraph{}
    Next, we notice that some people prefer to have Dark mode, and some prefer to have Light mode in their calculator. We’ll give an option to switch between both on Eternity. The given data also suggests that most of our interviewees have little to no computer literacy, so Eternity would come with a very easy and user-friendly interface and would also come with a User Guide. The User Guide would have documentation to explain how the functions work and how the basic structure of the calculator is laid out. Another important aspect of the calculator is its accuracy.

    \paragraph{}
    The interviews indicate that Jules and Shaunna expect the calculator to have accuracy to the 2nd decimal, whereas, Chad, Pierre, and Jim expect to have accuracy to the 4th decimal at least. Therefore, Eternity would have accuracy to the 4th decimal. Also, it is evident from the interviews that the potential users of the calculator would like to keep a history of the calculations and their results. We plan to add this feature in the calculator too. Some functions like MAD  requires users to input multiple values for each variable, so we asked the interviewees if they would like to enter these numbers all at once or one by one and we got mixed responses. Jules, Shauna and Jim would prefer one by one, Chad, Pierre and Evelyn would like to enter all the values at once, and Ricciardio would like the option to choose from both.

    \paragraph{}
    In addition, it is obvious from the interviews that the interviewees do not feel the need of the command line interface and therefore Eternity won’t have one. However, the potential users of this calculator would like to swap between the pages for the scientific functions and the normal arithmetic functions. Next question in the interview was about the budget and how much these interviewees were willing to pay for a calculator like Eternity. Evelyn and Ricciardio had a comparatively higher budget of \$50 and \$60-\$100 respectively. Whereas the other interviewees were ready to pay somewhere in between \$0 to \$40. The final question in the interview was if these interviewees had any other features they’d like to add in the calculator, and we received varied responses. Jules would like a percentage button in the calculator, Chad would appreciate it if the calculator was mobile and computer friendly and Shaunna would like a “Welcome” message on the calculator as that would please the customers at her shop into believing that she puts a lot of thought into her work environment. Pierre thinks that indication that the calculation is incorrect before pressing equal is important as it saves a lot of time. She and Jim also think that unit converters would be a great addition to Eternity. In addition to that, Jim would also like the calculator to have a longer battery life and be user friendly, durable and weatherproof. Evelyn thinks that a decimal to binary/decimal to hex converter would be a great feature to add in the calculator. And finally, Ricciardo would like to export all his calculations and data to files so he can share it with his colleagues.

    \paragraph{}
    In conclusion, the interviewees would like the Eternity to have all the basic functions along with some transcendental functions. They would also like to have a user guide along with the calculator and are willing to pay up to \$100 for the application.
