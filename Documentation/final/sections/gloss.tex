\section{Glossary}
    \begin{itemize}
        \item Arccos(x): The arccosine of x is when the cosine function of x is reversed when x is in between or equal to -1 and 1. When the cosine is set up to y is equal to x, the arccosine of x equals the reversed cosine function of x, which is equal to y. \cite{arccos}
        \begin{center}
            $arccos(x) = cos^{-1}(x) = y$
        \end{center}
        \item Euler’s Number: Denoted by e, Euler’s number is a constant in mathematics approximating to 2.71828. It should be noted that e is an irrational number.\cite{boundless}
        \item Irrational Number: A number is said to be irrational if it cannot be written as a ratio of two integers. An example of this is Euler’s number. \cite{wolfram}
        \item Integer: A number that is whole. \cite{merriam}
        \item Exponent: Given a number x and exponent y, the result given by the expression $x^y$ is the value of x multiplying itself y times (i.e. the result of $2^3$ gives $2 * 2 * 2 = 8$). It should be noted that the x and y variable can themselves be functions. The same principle applies to a constant to a power of x (i.e $2^{0.98x+3}$).
        \item Hyperbolic Sine: Hyperbolic function which traces a hyperbola. Calculated through formula: (where e is Euler’s Number) \cite{encyclopaedia}
        \begin{center}
            $sinh(x) =  \frac{e^x - e^{-x}} {2}$
        \end{center}
        \item Logarithmic function: It is the inverse of an exponential function. When using the log function without an explicitly defined base, the assumed base will be 10. For simplicity sake, the natural log known as $log10(x) = ln(x)$ will not be used within the documentation.\cite{dawkins}
        \begin{center}
            $f(x) = log_b(x) = b^{f(x)}x$
        \end{center}
        \item  Mean Absolute Deviation: It is the average distance between each data point and the mean for a specific dataset. \cite{khan}
        \item Standard Deviation: A measure of how spread out the data is compared to the average. Denoted with $\sigma$. \cite{students}
    \end{itemize}
