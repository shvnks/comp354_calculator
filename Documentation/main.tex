\documentclass[titlepage]{article}
\usepackage[utf8]{inputenc}

% used for enhanced justification
\usepackage{microtype}

% used for indexing
\usepackage{index}
\makeindex

% used for filling in blanks
\usepackage{blindtext}

% used for nicer headers
\usepackage{fancyhdr}

% math packages
\usepackage{amssymb}
\usepackage{mathtools}
\usepackage{graphicx}
\usepackage{natbib}
\usepackage{amsmath}
\usepackage{amsfonts}
\usepackage{amssymb}

\begin{document}
\title{Eternity}
\author{Tyler Shanks, Danny Roux, Andrei Serban, Vatsa Katrik Shah,\newpage William Robinson, Nicholas Simo, Arash Singh}
\date{June 2021}
\maketitle

\fancyfoot{\thepage}

\tableofcontents
\newpage

\section{Introduction}
    \subsection{Glossary}
        \blindtext[5]

    \subsection{Functions}
        \begin{enumerate}
            \item $\begin{aligned}[t]
                arccos(x)
            \end{aligned}$
                Worked on by Will

            \item $\begin{aligned}[t]
                ab^x
            \end{aligned}$
                Worked on by Andrei

            \item $\begin{aligned}[t]
                log\textsuperscript{b}(x)
            \end{aligned}$
                Worked on by Shanks

            \item $\begin{aligned}[t]
                \gamma(x)
            \end{aligned}$
                Worked on by No one worked on this function

            \item $\begin{aligned}[t]
                MAD
            \end{aligned}$
                Worked on by Vatsa

            \item $\begin{aligned}[t]
                \sigma
            \end{aligned}$
                Worked on by Nicholas

            \item $\begin{aligned}[t]
                sinh(x)
            \end{aligned}$
                Worked on by Arash

            \item $\begin{aligned}[t]
                x^y
            \end{aligned}$
                Worked on by Danny
        \end{enumerate}

    \subsection{Collaboration Patterns}
        Github + discord \newline Functions to team members

\section{Materials \& Methods}
        \subsection{Interviews}
            \subsubsection{Questions}
                \begin{enumerate}
                    \item Name? Age? Profession? Level of education?
                    \item What mathematics are invovlded in your day to day life?
                    \item What type of calculator is suitable for you?
                    \item What functions do you expect a calculator to have?
                    \item Would you enjoy having a customizable calculator? Such as different coloured buttons or UI.
                    \item Do you prefer having options such as "dark mode", "light mode" or having the option to choose?
                    \item Should there be documentation to explain how each function works within the calculator as well as how to interact with the UI?
                    \item How accurate do you expect the calculator to be? For example, to the 6th decimal place.
                    \item Would you like to be able to keep a history of calculations and their results?
                    \item For functions that require you to have several different values for each variable, would you rather they be entered one by one or all at once (in an array)?
                    \item Do you want to be able to use the calculator with a command line interface?
                    \item Would you like to swap between the pages for the scientific functions and the normal elementary arithmetic functions?
                    \item As for the display of the calculator, is it important for you to have a display that can display fractions and symbols as you would write them on paper?
                    \item How much would you be willing to pay for this calculator?
                    \item Are there any other features that you'd like this calculator to have?
                \end{enumerate}

            \subsubsection{Responses}
                \begin{itemize}
                    \item Julies Benson
                        \begin{enumerate}
                            \item My name is Jules Benson and I am 43 years old. I am a car salesman for Audi. My level of education is a bachelors degree in marketing.
                            \item Normally, I do math when I’m near my desk sitting down with a customer. Math in my job involves calculating the total cost of the vehicle, adding or subtracting vehicle options cost, and more importantly, calculating the monthly payment of the vehicles and the interests.
                            \item The most difficult calculation that I have to do during the whole day is calculating interests that involve a lot of percentages. So for my needs, a basic calculator, without any scientific function is sufficient.
                            \item I expect that a calculator has the ability to do basic calculations obviously. A percentage button would be helpful, that way I wouldn’t have to multiply everything by 100 each time I’m taking a percentage.
                            \item Not necessarily, It wouldn’t change much in my job, I’m looking for practicality over aesthetics.
                            \item My job is during the day, and the dealership is very bright as well, so I don’t really need a dark mode, but I wouldn’t mind it for those winter evenings when the sun goes down early.
                            \item Some documentation would be helpful just to know if there’s some things I previously didn’t know that a calculator can do.
                            \item I expect the calculator to be accurate to the 2nd decimal, which is needed when calculating interests on monthly payments.
                            \item Yes, that would be a good functionality. Sometimes customers are hesitating between different cars and different payment plans, so it will be practical to go back and see the previous calculation for comparison.
                            \item I don’t think I’ll be using that function but if anything, I would want them to be entered one by one.
                            \item I don’t know anything about programming, I am not very tech savvy. So, I don’t want to use it with a command line interface.
                            \item I will most likely be using arithmetic functions, but in some rare case that I end up using scientific functions, it would be a good idea to add a swap between pages just to separate the two functionalities.
                            \item Yes, that would be very helpful, since I wouldn’t have to use my brain too much on plugging in the calculator.
                            \item Around 20$
                            \item I mentioned that a percentage button would be helpful for me, so that’s one thing I would like the calculator to have
                            \item
                        \end{enumerate}
                    \item joe
                        \begin{enumerate}
                            \item
                            \item
                            \item
                            \item
                            \item
                            \item
                            \item
                            \item
                            \item
                            \item
                            \item
                            \item
                            \item
                            \item
                            \item
                            \item
                        \end{enumerate}
                    \item smo
                        \begin{enumerate}
                            \item
                            \item
                            \item
                            \item
                            \item
                            \item
                            \item
                            \item
                            \item
                            \item
                            \item
                            \item
                            \item
                            \item
                            \item
                            \item
                        \end{enumerate}
                    \item blow
                        \begin{enumerate}
                            \item
                            \item
                            \item
                            \item
                            \item
                            \item
                            \item
                            \item
                            \item
                            \item
                            \item
                            \item
                            \item
                            \item
                            \item
                            \item
                        \end{enumerate}
                    \item show
                        \begin{enumerate}
                            \item
                            \item
                            \item
                            \item
                            \item
                            \item
                            \item
                            \item
                            \item
                            \item
                            \item
                            \item
                            \item
                            \item
                            \item
                            \item
                        \end{enumerate}
                    \item hoe
                        \begin{enumerate}
                            \item
                            \item
                            \item
                            \item
                            \item
                            \item
                            \item
                            \item
                            \item
                            \item
                            \item
                            \item
                            \item
                            \item
                            \item
                            \item
                        \end{enumerate}
                    \item go
                        \begin{enumerate}
                            \item
                            \item
                            \item
                            \item
                            \item
                            \item
                            \item
                            \item
                            \item
                            \item
                            \item
                            \item
                            \item
                            \item
                            \item
                            \item
                        \end{enumerate}
                \end{itemize}

        \subsection{Analysis}
            \blindtext

        \subsection{Personas}
            \blindtext

% need to fix the references section
\section{References}
\bibliographystyle{IEEEtran}
\bibliography{references.bib}

\end{document}
